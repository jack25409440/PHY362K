\documentclass[12pt]{article}

\usepackage{graphicx}
\usepackage[margin=1.0in]{geometry}
\usepackage{amsmath}
\usepackage{cases}
\usepackage{amsfonts}
\usepackage{amssymb}
\usepackage{grffile}
\usepackage{setspace}

\setlength\parindent{0pt}

\author{Xiaohui Chen \\EID: xc2388}
\title{PHY 362K Homework 1}

\begin{document}
\maketitle
\begin{spacing}{2.0}

\section{} %1
\subsection*{(a)}
The normalized wave function satisfies the condition that $\int_{-\infty}^{\infty} |\Psi(x,0)|^2 dx =1$

Therefore, $\int_{-\infty}^{\infty} |\Psi(x,0)|^2 dx = A^2 \int_{-\infty}^{\infty} \left( 1+3 \sqrt{\frac{m\omega}{\hbar}}x\right)^2 e^{-\frac{m\omega}{\hbar} x^2} dx= \frac{11 \sqrt{\pi }}{2 \sqrt{\frac{m \omega }{\hbar }}} A^2 =1$

$A^2=\frac{2}{11}\sqrt{\frac{m\omega}{\pi \hbar}}$ and so $A=\sqrt{\frac{2}{11}}\left(\frac{m\omega}{\pi \hbar} \right)^{1/4}$

\subsection*{(b)}
From (a) we can get $\Psi(x,0)= \sqrt{\frac{2}{11}}\left(\frac{m\omega}{\pi \hbar} \right)^{1/4} \left( 1+3 \sqrt{\frac{m\omega}{\hbar}}x\right) e^{-\frac{m\omega}{2\hbar} x^2}= \sqrt{\frac{2}{11}}\left(\frac{m\omega}{\pi \hbar} \right)^{1/4} e^{-\frac{m\omega}{2\hbar} x^2} +\sqrt{\frac{18}{11}}\left(\frac{m\omega}{\pi \hbar} \right)^{1/4} \sqrt{\frac{m\omega}{\hbar}}x e^{-\frac{m\omega}{2\hbar} x^2}$

We know that $\psi_0(x)=\left(\frac{m\omega}{\pi \hbar} \right)^{1/4} e^{-\frac{m\omega}{2\hbar} x^2}$ and $\psi_1(x)= \left(\frac{m\omega}{\pi \hbar} \right)^{1/4} \sqrt{2} \sqrt{\frac{m\omega}{\hbar}} x e^{-\frac{m\omega}{2\hbar} x^2}$

Therefore we can know that $\Psi(x,0)$ can be represented as a linear combination of $\psi_0(x)$ and $\psi_1(x)$, which is $\Psi(x,0)= c_0\psi_0(x) + c_1\psi_1(x)$

$\therefore c_0=\sqrt{\frac{2}{11}}$ and $c_1=\sqrt{\frac{9}{11}}$

$\because E_n=\left( n+\frac{1}{2} \right)\hbar\omega$

$\therefore E_0=\frac{1}{2}\hbar\omega$ and $E_1=\frac{3}{2}\hbar\omega$

$\langle E \rangle= |c_0|^2 E_0+ |c_1|^2 E_1 = \frac{1}{11}\hbar\omega + \frac{27}{22}\hbar\omega = \frac{29}{22}\hbar\omega$

Therefore the particle has energy $\frac{1}{2}\hbar\omega$ with probability $\frac{2}{11}$ and has energy $\frac{3}{2}\hbar\omega$ with probability $\frac{9}{11}$. The expected value of energy is $\frac{29}{22}\hbar\omega$

\subsection*{(c)}

$$\Psi(x,t)= \frac{2}{11}\psi_0(x)e^{-\frac{iE_0}{\hbar}t} + \frac{9}{11}\psi_1(x)e^{-\frac{iE_1}{\hbar}t}= \frac{2}{11}\psi_0(x)e^{-\frac{i\omega}{2}t} + \frac{9}{11}\psi_1(x)e^{-\frac{i3\omega}{2}t}$$

where $\psi_0(x)=\left(\frac{m\omega}{\pi \hbar} \right)^{1/4} e^{-\frac{m\omega}{2\hbar} x^2}$ and $\psi_1(x)= \left(\frac{m\omega}{\pi \hbar} \right)^{1/4} \sqrt{2} \sqrt{\frac{m\omega}{\hbar}} x e^{-\frac{m\omega}{2\hbar} x^2}$

\section{} %2
We can let $|a_1\rangle=\left(
\begin{array}{c}
1\\
0\\
0
\end{array}
\right)$, $|a_2\rangle=\left(
\begin{array}{c}
0\\
1\\
0
\end{array}
\right)$ and $|a_3\rangle=\left(
\begin{array}{c}
0\\
0\\
1
\end{array}
\right)$

We also let $B=\left(
\begin{array}{ccc}
b_1 & b_2 & b_3\\
b_4 & b_5 & b_6\\
b_7 & b_8 & b_9
\end{array}
\right)$

$B|a_1\rangle= \left(
\begin{array}{ccc}
b_1 & b_2 & b_3\\
b_4 & b_5 & b_6\\
b_7 & b_8 & b_9
\end{array}
\right)\left(
\begin{array}{c}
1\\
0\\
0
\end{array}\right)= \left(
\begin{array}{c}
b_1\\
b_4\\
b_7
\end{array}\right)= \left(
\begin{array}{c}
0\\
1\\
1
\end{array}\right)$

$B|a_2\rangle= \left(
\begin{array}{ccc}
b_1 & b_2 & b_3\\
b_4 & b_5 & b_6\\
b_7 & b_8 & b_9
\end{array}
\right)\left(
\begin{array}{c}
0\\
1\\
0
\end{array}\right)= \left(
\begin{array}{c}
b_2\\
b_5\\
b_8
\end{array}\right)= \left(
\begin{array}{c}
2\\
1\\
-1
\end{array}\right)$

$B|a_3\rangle= \left(
\begin{array}{ccc}
b_1 & b_2 & b_3\\
b_4 & b_5 & b_6\\
b_7 & b_8 & b_9
\end{array}
\right)\left(
\begin{array}{c}
0\\
0\\
1
\end{array}\right)= \left(
\begin{array}{c}
b_3\\
b_6\\
b_9
\end{array}\right)= \left(
\begin{array}{c}
0\\
-1\\
0
\end{array}\right)$

$\therefore B=\left(
\begin{array}{ccc}
0 & 2 & 0\\
1 & 1 & -1\\
1 & -1 & 0
\end{array}
\right)$

\section{} %3
\subsection*{(a)}
From the equation of wave function for hydrogen, we get

$|2,1,1\rangle=-\sqrt{\left(\frac{2}{a}\right)^{3}\frac{1}{4*6^3}}e^{-\frac{r}{2a}}\left(\frac{r}{a}\right)
6\sqrt{\frac{3}{8\pi}}\sin(\theta)e^{i\phi}
=-\sqrt{\frac{1}{64\pi a^3}}\left(\frac{r}{a}\right)e^{-\frac{r}{2a}}\sin(\theta)e^{i\phi}$

Since $|2,1,-1\rangle$ only differs with $|2,1,1\rangle$ in the term $Y_l^m(\theta,\phi)$, we can know that

$|2,1,-1\rangle=\sqrt{\frac{1}{64\pi a^3}}\left(\frac{r}{a}\right)e^{-\frac{r}{2a}}\sin(\theta)e^{-i\phi}$

Therefore $\langle\vec{r}|\psi\rangle= -\sqrt{\frac{1}{80\pi a^3}}\left(\frac{r}{a}\right)e^{-\frac{r}{2a}}\sin(\theta)e^{i\phi} -\sqrt{\frac{1}{320\pi a^3}}\left(\frac{r}{a}\right)e^{-\frac{r}{2a}}\sin(\theta)e^{-i\phi}$

where $a\equiv 0.0529 nm$

\subsection*{(b)}
In both cases $\phi=0$. Therefore the wave function can be simplified to

$\langle\vec{r}|\psi\rangle= -\left(\sqrt{\frac{1}{320\pi a^3}}+\sqrt{\frac{1}{80\pi a^3}}\right) \left(\frac{r}{a}\right)e^{-\frac{r}{2a}}\sin(\theta) $

(1) When the coordinate is (0.1nm,0,0), $r=0.1$ and $\theta=0$

Therefore $\sin(\theta)=0$ and $\langle\vec{r}|\psi\rangle=0$

The volume $V=0.1*0.1*0.1 nm^3=10^{-30}m$

The probability density $\langle\psi|\vec{r}\rangle \langle\vec{r}|\psi\rangle=0$ and so the probability $Pr=\langle\psi|\vec{r}\rangle \langle\vec{r}|\psi\rangle*V=0$

(2) When the coordinate is (0,-0.1,0), $r=0.1$ and $\theta=\frac{3\pi}{2}$

Therefore $\sin(\theta)=-1$ and $\langle\vec{r}|\psi\rangle= -\left(\sqrt{\frac{1}{80\pi a^3}} +\sqrt{\frac{1}{320\pi a^3}} \right) \left(\frac{1000}{529}\right)e^{-\frac{5000}{529}}$

The probability density $\langle\psi|\vec{r}\rangle \langle\vec{r}|\psi\rangle= \left(\sqrt{\frac{1}{80\pi a^3}} +\sqrt{\frac{1}{320\pi a^3}} \right)^2 e^{-\frac{0.1}{0.0529}}*(\frac{0.1}{0.0529})^2 \approx 3.264*10^{28}$

Therefore the probability $Pr=\langle\psi|\vec{r}\rangle \langle\vec{r}|\psi\rangle*V \approx = 0.0326$

\subsection*{(c)}
The eigenvalue of $L_z$ operator is $\hbar m$

We know that $|\psi\rangle=\frac{2}{\sqrt{5}} |2,1,1\rangle - \sqrt{\frac{1}{5}}|2,1,-1\rangle$

Therefore, the measurement of $L_z$ is $\hbar$ with probability $\frac{4}{5}$ and $-\hbar$ with probability $\frac{1}{5}$

\section{} %4

i. $|\gamma \rangle= |\psi\rangle + i|\phi\rangle= \left(
\begin{array}{c}
-i\\
i
\end{array}
\right)+ i\left(
\begin{array}{c}
i\\
1
\end{array}
\right)= \left(
\begin{array}{c}
-i-1\\
i+i
\end{array}
\right)= \left(
\begin{array}{c}
-i-1\\
2i
\end{array}
\right)$

ii. $\langle \psi|=\left(
\begin{array}{cc}
i & -i
\end{array}
\right)$

iii. $\langle \gamma|= \left(
\begin{array}{cc}
i+1 & -2i
\end{array}
\right)$

iv. $\langle \psi|\phi \rangle= \left(
\begin{array}{cc}
i & -i
\end{array}
\right)\left(
\begin{array}{c}
i\\
1
\end{array}
\right)= -1-i$

v. $\langle \phi|\psi \rangle= \left(
\begin{array}{cc}
-i & 1
\end{array}
\right)\left(
\begin{array}{c}
-i\\
i
\end{array}
\right)=-1+i$

vi. $A^{+}= \left(
\begin{array}{cc}
0 & -i\\
1 & 0
\end{array}
\right)$

vii. $A|\psi\rangle= \left(
\begin{array}{cc}
0 & 1\\
i & 0
\end{array}
\right)\left(\begin{array}{c}
-i\\
i
\end{array}
\right)=\left(\begin{array}{c}
i\\
1
\end{array}
\right)$

viii. $\langle \psi|A= \left(
\begin{array}{cc}
i & -i
\end{array}
\right)\left(
\begin{array}{cc}
0 & 1\\
i & 0
\end{array}
\right)=\left(
\begin{array}{cc}
1 & i
\end{array}
\right)$

ix. $\langle \psi|A^{+} = \left(
\begin{array}{cc}
i & -i
\end{array}
\right)\left(
\begin{array}{cc}
0 & -i\\
1 & 0
\end{array}
\right)= \left(
\begin{array}{cc}
-i & 1
\end{array}
\right)$

x. $(AB)^{+}= \left(\left(
\begin{array}{cc}
0 & 1\\
i & 0
\end{array}
\right)\left(
\begin{array}{cc}
1 & 0\\
0 & -1
\end{array}
\right) \right)^{+}= \left(
\begin{array}{cc}
0 & -1\\
i & 0
\end{array}
\right)^{+}= \left(
\begin{array}{cc}
0 & -i\\
-1 & 0
\end{array}
\right)$

xi. $\langle \psi|A|\phi \rangle= \left(
\begin{array}{cc}
i & -i
\end{array}
\right)\left(
\begin{array}{cc}
0 & 1\\
i & 0
\end{array}
\right)\left(
\begin{array}{c}
i\\
1
\end{array}
\right)=2i$

xii. $\langle \psi|A^{+}|\phi \rangle=\left(
\begin{array}{cc}
i & -i
\end{array}
\right)\left(
\begin{array}{cc}
0 & -i\\
1 & 0
\end{array}
\right)\left(
\begin{array}{c}
i\\
1
\end{array}
\right)=2$

xiii. $\langle \phi|A|\psi \rangle=\left(
\begin{array}{cc}
-i & 1
\end{array}
\right)\left(
\begin{array}{cc}
0 & 1\\
i & 0
\end{array}
\right)\left(
\begin{array}{c}
-i\\
i
\end{array}
\right)=2$

xiv. $|\phi\rangle \langle \psi|= \left(
\begin{array}{c}
i\\
1
\end{array}
\right)\left(
\begin{array}{cc}
i & -i
\end{array}
\right)=\left(
\begin{array}{cc}
 -1 & 1 \\
 i & -i \\
\end{array}
\right)$

xv. $|\phi\rangle \langle \psi|B=\left(
\begin{array}{c}
i\\
1
\end{array}
\right)\left(
\begin{array}{cc}
i & -i
\end{array}
\right)\left(
\begin{array}{cc}
 1 & 0 \\
 0 & -1 \\
\end{array}
\right)= \left(
\begin{array}{cc}
 -1 & -1 \\
 i & i \\
\end{array}
\right)$

\section{} %5
\subsection*{(a)}
Normalizing $\Phi(p,0)$ means $\int_{-\infty}^{\infty} |\Phi(p,0)|^2 dp=1$

$\therefore \int_{-\infty}^{\infty} |\Phi(p,0)|^2 dp= N^2\int_{-\infty}^{\infty}e^{-\frac{2a}{\hbar}|p|}= N^2\frac{\hbar}{a}=1$

Therefore $N=\sqrt{\frac{a}{\hbar}}$

\subsection*{(b)}
According to the Fourier Transformation,

$\Psi(x,0) = \frac{1}{\sqrt{2\pi \hbar}} \int\limits_{-\infty}^{\infty} \Phi(p,0) e^{i\frac{p}{\hbar}x} dp$

$\Phi(p,0)=\sqrt{\frac{a}{\hbar }} e^{-\frac{a \left| p\right| }{\hbar }}$

Therefore, $\Psi(x,0)= \sqrt{\frac{2a^3}{\pi}}\frac{1}{a^2+x^2}$

\subsection*{(c)}
$\langle x \rangle=\int_{-\infty}^{\infty} x|\Psi(x,0)|^2 dx=0$

$\langle x^2 \rangle=\int_{-\infty}^{\infty} x^2|\Psi(x,0)|^2 dx=\frac{a^2}{2}$

$\Delta x_{rms}=\sqrt{\langle x^2 \rangle- \langle x \rangle^2}= \frac{a}{\sqrt{2}}$

$\langle p \rangle=\int_{-\infty}^{\infty} p|\Phi(p,0)|^2 dx=0$

$\langle p^2 \rangle=\int_{-\infty}^{\infty} p^2|\Phi(p,0)|^2 dx=\frac{\hbar ^2}{2 a^2}$

$\Delta p_{rms}=\sqrt{\langle p^2 \rangle- \langle p \rangle^2}= \frac{\hbar}{\sqrt{2}a}$

$\therefore \Delta x_{rms} \Delta p_{rms}= \frac{\hbar}{2}\ge \frac{\hbar}{2}$

Hence the uncertainty principle preserves

\subsection*{(d)}
Yes, the momentum space wave function changes with time

The Fourier Transform gives:

$\Phi(p,0) = \frac{1}{\sqrt{2\pi \hbar}} \int\limits_{-\infty}^{\infty} \psi(x) e^{-i\frac{p}{\hbar}x} dx$

$\therefore \Phi(p,t) = \frac{1}{\sqrt{2\pi \hbar}} \int\limits_{-\infty}^{\infty} \psi(x) e^{-i(\frac{p}{\hbar}x+ \frac{p^2t}{2m\hbar})} dx$


The probability does not change with time since the probability density does not integrate with time %explanation

\subsection*{(e)}
Yes, the position space wave function changes with time

The Inverse Fourier Transform gives:

$\Psi(x,0) = \frac{1}{\sqrt{2\pi \hbar}} \int\limits_{-\infty}^{\infty} \Phi(p,0) e^{i\frac{p}{\hbar}x} dp$

$\therefore \Psi(x,t) = \frac{1}{\sqrt{2\pi \hbar}} \int\limits_{-\infty}^{\infty} \Phi(p,0) e^{i(\frac{p}{\hbar}x- \frac{p^2t}{2m\hbar})} dp$

The probability does not change with time since the probability density does not integrate with time %explanation
\end{spacing}
\end{document} 