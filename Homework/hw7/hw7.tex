\documentclass[12pt]{article}

\usepackage{graphicx}
\usepackage[margin=1.0in]{geometry}
\usepackage{amsmath}
\usepackage{cases}
\usepackage{amsfonts}
\usepackage{amssymb}
\usepackage{grffile}
\usepackage{setspace}
\usepackage{listings}

\setlength\parindent{0pt}

\author{Xiaohui Chen \\EID: xc2388}
\title{PHY 362K Homework 7}

\begin{document}
\maketitle

\begin{spacing}{2.0}

\section{} %1

Since the particle is in infinite square well, the engergy is $E_n= \frac{n^2 \pi^2 \hbar^2}{2ma^2}$

$\psi_n(x)= \sqrt{\frac{2}{a}} \sin \left( \frac{n\pi}{a}x  \right)$

Since the transition is from state $n=1$ to state $n=2$, we get $\omega_{21} = \frac{E_2-E_1}{\hbar} = \frac{3\pi^2 \hbar}{2ma^2}$

$H'_{21}(t) = \langle 2| H'|1 \rangle  = \int_{0}^{a} \psi_2^*(x) Ax^2e^{-\gamma t} \psi_1(x) dx = -\frac{16 a^2 A e^{-t \gamma }}{9 \pi ^2}  $ (Using Mathematica)

$\therefore c_2 = -\frac{i}{\hbar} \int_{0}^{T} e^{i\omega_{21}t } H'_{21}(t) dt = \frac{32 i a^4 A \left(1-e^{-T \gamma +\frac{3 i \pi ^2 T \hbar }{2 a^2 m}}\right) m}{9 \pi ^2 \hbar  \left(2 a^2 m \gamma -3 i \pi ^2 \hbar \right)} $

$\therefore P_{1 \rightarrow 2} = |c_2|^2 = \frac{1024 a^8 A^2 m^2}{81\pi^4 \hbar^2(4a^4 m^2\gamma^2 + 9\pi^4\hbar^2)} \left(1-e^{-T \gamma +\frac{3 i \pi ^2 T \hbar }{2 a^2 m}}\right) \left(1-e^{-T \gamma -\frac{3 i \pi ^2 T \hbar }{2 a^2 m}}\right)$

When $T \rightarrow \infty$, $ \left(1-e^{-T \gamma +\frac{3 i \pi ^2 T \hbar }{2 a^2 m}}\right) \rightarrow 0 $ and $ \left(1-e^{-T \gamma -\frac{3 i \pi ^2 T \hbar }{2 a^2 m}}\right) \rightarrow 0 $

Therefore in this case $P_{1 \rightarrow 2} = \frac{1024 a^8 A^2 m^2}{81\pi^4 \hbar^2(4a^4 m^2\gamma^2 + 9\pi^4\hbar^2)}$

\section{} %2

\subsection*{(a)}

\subsection*{(b)}

\subsection*{(c)}

\section{} %3

\subsection*{(a)}

\subsection*{(b)}

\section{} %4

\section{} %5

\subsection*{(a)}

\subsection*{(b)}


\end{spacing}
\end{document}
