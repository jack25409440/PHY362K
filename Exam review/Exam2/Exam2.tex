\documentclass[12pt]{article}

\usepackage{graphicx}
\usepackage[margin=1.0in]{geometry}
\usepackage{amsmath}
\usepackage{cases}
\usepackage{amsfonts}
\usepackage{amssymb}
\usepackage{grffile}
\usepackage{setspace}
\usepackage{listings}

\setlength\parindent{0pt}

\author{Xiaohui Chen}
\title{PHY 362K Midterm 2 Review Note}
\date{03/10/2015}

\begin{document}
\maketitle

\begin{spacing}{2.0}

\section{Normal Zeeman Effect}

Bohr magneton: $\mu_B = \frac{e\hbar}{2m_e}$

The levels with $m_l=0,+1,-1$ have energies $E_P^{(0)}$, $E_P^{(0)}+ \mu_B B$, $E_P^{(0)}- \mu_B B$

The Bohr transition frequencies of these levels to a lower $s$ level are just $\omega_0$, $\omega_0+ \frac{\mu_B B}{\hbar}$, $\omega_0- \frac{\mu_B B}{\hbar}$ respectively

The normal Zeeman effect does occur in real atoms when all electron spins are paired, such that the total spin of all electrons is zero

Fine structure constant: $\alpha= \frac{e^2}{4\pi \epsilon_0 \hbar c} \approx \frac{1}{137.04}$

Physics spectroscopic notation: $n^{2s+1}l_j$

\section{Stark Effect in Hydrogen}

$\vec{\mathcal{E}}= \mathcal{E}z$

$H'=e\mathcal{E}z$

$E_1^{(1)}= \langle \psi_{100}^{(0)}| H'| \psi_{100}^{(0)}\rangle= e\mathcal{E}\langle \psi_{100}^{(0)}|z| \psi_{100}^{(0)}\rangle = 0$

$|\psi_{100}^{(1)} \rangle= \sum\limits_{(nlm)\ne 100} \frac{\langle \psi_{nlm}^{(0)}| H'| \psi_{100}^{(0)}\rangle}{E_1^{(0)}-E_n^{(0)}}$

$\psi_{nlm}^{(0)}|z| \psi_{n'l'm'}^{(0)}\rangle = 0$ unless

(1) $m=m'$ and 

(2) $l=l'+1$ and $l=l'-1$

Energy Shift:

$E_1^{(2)}= \frac{e\mathcal{E}^2}{E_1^{(0)}} \sum\limits_{n=2}^{\infty} \frac{|\psi_{n10}^{(0)}|z| \psi_{100}^{(0)}\rangle|^2}{1-\frac{1}{n^2}}= -2*4\pi\epsilon_0 a_0 \mathcal{E}^2 \sum\limits_{n=2}^{\infty} \frac{|\psi_{n10}^{(0)}|z| \psi_{100}^{(0)}\rangle|^2}{1-\frac{1}{n^2}}$

\section{Electron Spin}

$[S_x,S_y]= i\hbar S_z$, $[S_y,S_z]= i\hbar S_x$, $[S_z,S_x]= i\hbar S_y$

$S^2 |s\ m\rangle= \hbar^2s(s+1)|s\ m\rangle$

$S_z |s\ m\rangle = \hbar m|s\ m\rangle$



\subsection{Raising and Lowering Operators}

$L_{\pm} |l\ m_l \rangle = \hbar \sqrt{l(l+1)-m_l(m_l \pm 1)} |l\ (m_l \pm 1) \rangle$

$S_{\pm} |s\ m_s \rangle = \hbar \sqrt{s(s+1)-m_s(m_s \pm 1)} |s\ (m_s\pm 1) \rangle$

$\because \vec{j}= \vec{l}+ \vec{s}$

$\therefore J_{\pm}= L_{\pm} + S_{\pm}$

$J_{\pm} |j\ m_j \rangle = \hbar \sqrt{j(j+1)-m_j(m_j \pm 1)} |s\ (m_j \pm 1) \rangle$

\section{State of Hydrogen Atom including Spin}

$\{ |n l m_l m_s \rangle \}$ are also eigenvectors of $H_z$

$H_z |n l m_l m_s \rangle = \mu_B B (m_l+ 2m_s)|n l m_l m_s \rangle$

$\{ |n l j m_j \rangle \}$ are also eigenvectors of $H_{fs}$

$H_{fs} |n l j m_j \rangle = - |E_n^{(0)}| \left( \frac{\alpha}{n} \right)^2 \left[ \frac{n}{j+\frac{1}{2}} - \frac{3}{4n^4} \right]$

Neither set are eigenvectors of both $H_{fs}$ and $H_z$

Need to use degenerate perturbation theory with $H' = H_{fs}+ H_{z}$ for accurate description

%Good quantum numbers

\section{Relativistic Effects of the Hydrogen Atom}

The energy of a free particle is $E=\sqrt{(m_e c^2)^2 + (pc)^2}$. Therefore, $i\hbar \frac{\partial \psi}{\partial t} = \sqrt{m_e^2 c^4 - \hbar^2 c^2 \vec{\bigtriangledown}^2} \psi$

Alternatively, we should write $H^2= m_e^2 c^4 + p^2 c^2$, then we write $-\hbar^2 \frac{\partial^2 \psi}{\partial t^2} = (m_e^2 c^4 - \hbar^2 c^2 \vec{\bigtriangledown}^2) \psi$

The atom is placed in uniform applied fields $\vec{\mathcal{E}}= \mathcal{E} \hat{z}$ and $\vec{B}= B \hat{z}$. Then the Hamiltonian is $H= H_0 - \vec{\mu} \cdot \vec{B} + \frac{e^2 B^2}{8m_e} \left( x^2 + y^2 \right) + e\mathcal{E} z + H_{fs}$

$H_0= \frac{\vec{p}^2}{2m_e} - \frac{e^2}{4\pi \epsilon_0 r}$

$\vec{\mu} = \vec{\mu_l} + \vec{\mu_s} = -\mu_B \left( \frac{\vec{l}+ g_e \vec{s}}{\hbar} \right)$

$g_e \approx 2$

\section{Fine Structure of the Hydrogen Atom}

$H_{fs}= H_{kin} + H_{so} + H_D$

kinetic term: $H_{kin}= -\frac{\vec{p}^4}{8m_e^3 c^2}$

spin-orbit term: $H_{so}= \left( \frac{e^2}{4\pi \epsilon_0} \right) \left( \frac{1}{2m_e^2 c^2} \right) \frac{\vec{l} \cdot \vec{s}}{r^3}$

Darwin term: $H_D= \frac{\hbar^2}{8m_e^2 c^2} 4\pi \left( \frac{e^2}{4\pi\epsilon_0} \right) \delta^3(\vec{r})$

$\langle H_{kin} \rangle = E_n^{(0)} \frac{\alpha^2}{n^2}$

$\langle H_{so} \rangle = \left( \frac{e^2}{4\pi\epsilon_0} \right) \left( \frac{1}{2m_e^2 c^2} \right) \langle \vec{l} \cdot \vec{s} \rangle \langle \frac{1}{r^3} \rangle $

$\langle H_{D} \rangle= \frac{\alpha^2 m_e c^2}{2n^2} \frac{\alpha^2}{n} \delta_{l0}= -E_n^{(0)} \frac{\alpha^2}{n} \delta_{l0}$

$\langle n,l,j,m_j|H_{fs}| n,l,j,m_j \rangle = \left( -\frac{1}{2n^2} \alpha^2 m_e c^2 \right) \left( \frac{\alpha^2}{n^2} \right) \left( \frac{n}{j+\frac{1}{2}}- \frac{3}{4} \right)$

$E_n^{(0)}= \left( -\frac{1}{2n^2} \alpha^2 m_e c^2 \right)$

\section{Abnormal Zeeman Effect}

\subsection{High-Field Limit}

$|\langle H_z \rangle| >> |\langle H_{fs} \rangle|$

$H= H_0 +H_z + H_{fs}$

The first two terms are large while the last term is small

$E_{m_l m_s} \approx E_n^{(0)} + \mu_B (m_l +2m_s) + \langle n l m_l m_s |H_{fs}| n l m_l m_s \rangle$

\subsection{Low-Field Limit}

$|\langle H_z \rangle| << |\langle H_{fs} \rangle|$

$H= H_0 +H_{fs} + H_{z}$

The first two terms are large while the last term is small

$E_{j m_j} \approx E_n^{(0)} - \frac{\alpha^2 |E_n^{(0)}|}{n^4} \left[ \frac{n}{j+\frac{1}{2}} - \frac{3}{4n^4} \right] + \frac{\mu_B B}{\hbar} \langle n l j m_j |l_z+2s_z| n l j m_j \rangle$

$\langle H_z \rangle = g_j \mu_B B m_j$

Lande g-factor: $g_j = \left[ 1+ \frac{j(j+1)+s(s+1)-l(l+1)}{2j(j+1)} \right]$

\subsection{Good Quantum Numbers}

correspond to observables that approximately commute with H

eigenvectors of H will be labelled by the good quantum numbers

\section{Basic Properties of Nuclei}

$m_p=1.673*10^{-27} kg$

$m_n = 1.675*10^{-27} kg$

$d_{nucl} \approx 2.5 A^{\frac{1}{3}} fm$

$1 fm = 10^{-15} m$

I is integer if \#protrons+\#electrons=even

I is half integer if \#protrons+\#electrons=odd

\section{Hyperfine Structure}

The nuclear spin eigenstate is given by $|I,M_i \rangle$, where $M_I = -I, \ldots, I$. The eigenvalue equations are

$\vec{I}^2 |I ,M_I \rangle = \hbar^2 I(I+1) |I,M_I \rangle$

$I_z |I ,M_I \rangle= \hbar M_I |I ,M_I \rangle$

The nuclear magnetic dipole moment is $\vec{\mu_I} = g_I \mu_n \frac{\vec{I}}{\hbar}$

$\mu_n = \frac{e\hbar}{2m_p}= 5.051 *10^{-27} J/T$ is the nuclear magneton

$H_{hf}= -\frac{\mu_0}{4\pi} \left\{ \frac{2\mu_B}{\hbar r^3} \vec{l} \cdot \vec{\mu_I} - \frac{1}{r^3} \left[ \vec{\mu_e} \cdot \vec{\mu_I} - 3(\vec{\mu_I} \cdot \hat{r})(\vec{\mu_e} \cdot \hat{r}) \right] + \frac{8\pi}{3} \vec{\mu_e} \cdot \vec{\mu_I} \delta^3(\vec{r}) \right\}$

When the electron is in ground state, it is reduced to Fermi Hamiltonian

$H_{Fermi}= -\frac{8\pi}{3} \left( \frac{\mu_0}{4\pi} \right) \vec{\mu_e} \cdot \vec{\mu_I} \delta^3(\vec{r})= \frac{8\pi}{3} \left( \frac{\mu_0}{4\pi} \right) g_e \mu_B g_I \mu_n \frac{\vec{I} \cdot \vec{s}}{\hbar^2} \delta^3(\vec{r})= A \frac{\vec{I} \cdot \vec{s}}{\hbar^2}$

$A=\frac{8\pi}{3} \left( \frac{\mu_0}{4\pi} \right) g_e \mu_B g_I \mu_n |\psi(0)|^2$

\end{spacing}
\end{document} 